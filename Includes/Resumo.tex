% Resumo em l�ngua vern�cula
\begin{center}
	{\Large{\textbf{\Title}}}
\end{center}

\vspace{1cm}

\begin{flushright}
	Autor: \Author\\
	Orientadora: \Advisor\\
	%Co-orientador: \Coadvisor
\end{flushright}

\vspace{1cm}

\begin{center}
	\Large{\textsc{\textbf{Resumo}}}
\end{center}

\noindent Jogos de estrat�gia em tempo real (RTS) apresentam muitos desafios para a cria��o de intelig�ncias artificiais. Um destes desafios � criar um plano de a��es efetivo dentro de um dado contexto. Um dos jogos utilizados como plataforma para cria��o de intelig�ncias artificiais competitivas � o StarCraft. Tais intelig�ncias artificiais para jogos t�m dificuldade em se adaptar e criar bons planos para combater a estrat�gia inimiga. Neste trabalho, um novo modelo de escalonamento de tarefas � proposto para os problemas de planejamento em jogos RTS. Este modelo considera eventos c�clicos e consiste em resolver um problema multiobjetivo que satisfaz restri��es impostas pelo jogo. S�o considerados recursos, tarefas e eventos c�clicos que traduzem as caracter�sticas do jogo em um caso do problema. O estado inicial do jogo cont�m as informa��es sobre os recursos, tarefas incompletas e eventos ativos. A estrat�gia define quais recursos maximizar ou minimizar e quais restri��es s�o aplicadas aos recursos, bem como o horizonte de projeto. S�o investigados quatro otimizadores multiobjetivo: NSGA-II e sua variante focada em joelhos, GRASP e Col�nia de Formigas. Experimentos com casos baseados em problemas reais de Starcraft s�o reportados. Nestes experimentos, o NSGA-II mostrou um desempenho superior aos outros otimizadores.

\noindent\textit{Palavras-chave}: Modelos de Otimiza��o, Otimiza��o Multiobjetivo, Jogos de Estrat�gia em Tempo Real, Planejamento de Projeto.